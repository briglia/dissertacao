\chapter{Ap�ndices}

\section{Ap�ndice - A}

Algoritmos utilizados em \cite{Mlin:06}, para inicia��o dos pesos sin�pticos dos neur�nios da grade, encontrar o BMU e c�lculo da dist�ncia euclidiana. Esses algoritmos s�o parte do treinamento das redes neurais e gera��o dos SOM's.

\begin{program}
\begin{verbatim}

void init_grid(struct som_node * grids[][GRIDS_YSIZE]) {
    int i, j, k;
    srand(time(0));
    for (i=0; i<GRIDS_XSIZE; i++) {
        for (j=0; j<GRIDS_YSIZE; j++) {
            grids[i][j] = (struct som_node *)
                    malloc(sizeof(struct som_node));
            grids[i][j]->xp = i;
            grids[i][j]->yp = j;
            for (k=0; k<WEIGHTS_SIZE; k++) {
                grids[i][j]->weights[k] = rand_float();
            }
        }
    }
}
\end{verbatim}
  \caption{Inicializa��o dos pesos sin�pticos dos neur�nios da grade.}
  \label{tab:init_som}
\end{program}

\begin{program}
\begin{verbatim}

struct som_node * get_bmu(double input_vector[],
                      int len_input,
                      struct som_node * grids[][GRIDS_YSIZE]
                      ) {
    struct som_node *bmu = grids[0][0];
    double best_dist = euclidean_dist(input_vector, 
                                      bmu->weights, 
                                      len_input);
    double new_dist;
    int i, j;
    for (i=0; i<GRIDS_XSIZE; i++) {
        for (j=0; j<GRIDS_YSIZE; j++) {
            new_dist = euclidean_dist(input_vector,
                                      grids[i][j]->weights,
                                      len_input);
            if (new_dist < best_dist) {
                bmu = grids[i][j];
                best_dist = new_dist;
            }
        }
    }
    return bmu;
}
\end{verbatim}
  \caption{Pesquisa o neur�nio vencedor (BMU) durante o processo competitivo.}
  \label{tab:bmu}	
\end{program}

\begin{program}
\begin{verbatim}

double euclidean_dist(double *input, double *weights, int len) {
    double summation = 0;
    double temp;
    int i;
    for (i=0; i<len; i++) {
        temp = (input[i]-weights[i]) * (input[i]-weights[i]);
        summation += temp;
    }
    return summation;
}
\end{verbatim}
\caption{C�lculo da dist�ncia Euclidiana.}
    \label{tab:euclidean}
\end{program}

\section{Ap�ndice - B}

\begin{program}
 \begin{verbatim}
int read_trained_som(char filename[], struct som_node * grids[][GRIDS_YSIZE]) {
    FILE * fp;
    ssize_t bytes_read;
    size_t len = 0;
    char * line = NULL;
    char * input[WEIGHTS_SIZE];
    int i, j;

    fp = fopen(filename, "r");

    if (fp == NULL) {
        return 0; 
    }

    for (i=0; i<WEIGHTS_SIZE; i++) 
        input[i] = (char *)malloc(10);

    while((bytes_read = getline(&line, &len, fp)) != -1) {
        sscanf(line, "(%d, %d) %s %s %s",
                     &i, &j, input[0], input[1], input[2]);

        grids[i][j] = (struct som_node *)
                     malloc(sizeof(struct som_node));
        grids[i][j]->xp = i;
        grids[i][j]->yp = j;
        grids[i][j]->weights[0] = strtod(input[0], NULL);
        grids[i][j]->weights[1] = strtod(input[1], NULL);
        grids[i][j]->weights[2] = strtod(input[2], NULL);
    }

    for (i=0; i<WEIGHTS_SIZE; i++)
        free(input[i]);
		
        if (line)
            free(line);
	
        if (fclose(fp)) {
            fprintf(stderr,
                         "error closing file %s: %s\n",
                         filename, strerror(errno));
            exit(EXIT_FAILURE);
        }
        return 1; 
}

 \end{verbatim}
 \caption{Fun��o que l� os dados do saved\_som.txt.}
\end{program}
